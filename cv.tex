\documentclass{resume} % Use the custom resume.cls style
\usepackage{fontawesome}
\usepackage[left=0.4 in,top=0.4in,right=0.4 in,bottom=0.4in]{geometry} % Document margins
\newcommand{\tab}[1]{\hspace{.2667\textwidth}\rlap{#1}} 
\newcommand{\itab}[1]{\hspace{0em}\rlap{#1}}
\name{Pol Ros Domènech} % Your name
% You can merge both of these into a single line, if you do not have a website.
\address{23 years \\ +34 638 92 69 61 \\ Girona / Barcelona} 
\address{\faEnvelopeSquare \hspace{0.25em} \href{polrosdomenech@gmail.com}{polrosdomenech@gmail.com} \hspace{0.25em} \faLinkedinSquare \hspace{0.25em}  \href{https://www.linkedin.com/in/pol-ros-domenech-436211187/}{pol-ros-domenech}  \hspace{0.25em} \faGithub \hspace{0.25em} \href{https://github.com/pololo300}{pololo300}  }%

\begin{document}

\begin{rSection}{OBJECTIVE}

 Mathematics and Computer Science student with a strong passion for learning, problem-solving, and analytical thinking, particularly in the fields of algorithms and complexity. As I near the completion of both degrees and reflect on my experiences working in two companies, I have discovered a deep desire to further my studies and advance my understanding in these areas.

\end{rSection}


\begin{rSection}{Education}
 {\bf Basic Education and Baccalaureate}, \href{https://www.bell-lloc.org/en/}{Bell-lloc del Pla} \hfill {2019}\\
 Technological modality together with the \href{https://www.ibo.org/}{International Baccalaureate Diploma (IB)}.

 {\bf Mathematics}, \href{https://web.ub.edu/en/home}{Universitat de Barcelona} \hfill {2019 - 2025}\\
 \textit{Pending subjects}: final thesis 

 {\bf Computer Science}, \href{https://www.upc.edu/en?set_language=en}{Universitat Politècnica de Catalunya} \hfill {2020 - 2025}\\
 Specialize in computation. \textit{Pending subjects}: final thesis.
\end{rSection}

\begin{rSection}{EXPERIENCE}
 {\bf \href{https://www.geonumerics.es/}{GeoNumerics}}, Castelldefels \hfill {July - October 2023} \\
 In a small company of expert a small company of expert geomaticians,  my involvement was in a European project where my responsibility was ensuring compatibility of the company’s libraries for Android platforms. With no prior expertise available in the company, I embarked on a self-directed learning journey to accomplish this task.

 {\bf \href{https://serimag.com/en/}{Serimag}}, Sant Joan Despí \hfill {current, since July 2024} \\
 Automating document data extraction processes by leveraging advanced technologies, including neural networks, OCR (Optical Character Recognition) inside synchronous architectures. My role was being the main developer of one project for ING.

\textbf{\href{https://hackupc.com/}{HackUpc}}, Barcelona \hfill  May 2021 and May 2022 \\
 Participant on the hackaton where I won the University best app, for developing enFIBat together with 2 friends, using react native. In HackUpc 2022 our team developed a website.
\end{rSection}

\begin{rSection}{GENERAL}
\textbf{LANGUAGES:} Catalan (Native), Spanish (Bilingual), English. 

\textbf{SKILLS:}
I don't have many references to be able to say what level I have, but the languages I have been using these years are (in order of knowledge): C, Python, C++, Rust, Java, Haskell, Open GL, SQL, Prolog, Wolfram Mathematica, R. I am also used to working with \LaTeX. 


From 2014 to 2020, I was a federated handball player, an experience that taught me discipline, teamwork, and resilience. Beyond sports, I am passionate about outdoor activities such as hiking, skiing, and exploring the mountains.

Since 2018, I have sought diverse work experiences, including positions at Hospital Santa Caterina, IAS, El Collel, and Residencia Salle-Bonanova, which allowed me to develop adaptability and interpersonal skills in various professional environments.

I began studying Computer Science at UPC one year after embarking on my Mathematics degree at UB. Balancing studies at two different universities was a significant challenge, but it strengthened my time management and problem-solving abilities. Additionally, I have actively participated in faculty activities, including joining a castellers (traditional human towers from Catalonia) team and contributing to the vibrant university community.

\end{rSection} 
\end{document}
