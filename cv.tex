\documentclass{resume} % Use the custom resume.cls style

\usepackage[left=0.4 in,top=0.4in,right=0.4 in,bottom=0.4in]{geometry} % Document margins
\newcommand{\tab}[1]{\hspace{.2667\textwidth}\rlap{#1}} 
\newcommand{\itab}[1]{\hspace{0em}\rlap{#1}}
\name{Pol Ros Domènech} % Your name
% You can merge both of these into a single line, if you do not have a website.
\address{23 years \\ 638 92 69 61 \\ Girona / Barcelona} 
\address{\href{polrosdomenech@gmail.com}{polrosdomenech@gmail.com} \\ \href{https://www.linkedin.com/in/pol-ros-domenech-436211187/}{linkedin.com}  \\ \href{https://github.com/pololo300}{github.com/pololo300}  }%

\begin{document}

%-----------------------------------------------http://www.faangpath.com/-----------------------------------------
%	OBJECTIVE
%----------------------------------------------------------------------------------------

\begin{rSection}{OBJECTIVE}

{fancy presentation, something like; Mathematics and Computer Science student with a strong enthusiasm for learning, mastering and deploying technical
and analytical abilities. I’m interested in algorithms and computation, an area in which I want to gain experience
and become proficient.}


\end{rSection}

%----------------------------------------------------------------------------------------
%	EXPERIENCE SECTION
%----------------------------------------------------------------------------------------

\begin{rSection}{EXPERIENCE}

{\bf GeoNumerics}, Castelldefels \hfill {July - October, 2023}\\
In a small company of expert a small company of expert geomaticians,  my involvement was in a European project where my responsibility was ensuring compatibility of the company’s libraries for Android platforms. With no prior expertise available in the company, I embarked on a self-directed learning journey to accomplish this task.

{\bf Serimag}, Barcelona \hfill {July 2024}\\
Ive been working ...

\end{rSection}
%----------------------------------------------------------------------------------------
%	EDUCATION SECTION
%----------------------------------------------------------------------------------------

\begin{rSection}{Education}

{\bf Basic Education and Baccalaureate}, Bell-lloc del Pla \hfill {2019}\\
Technological modality together with the International Baccalaureate Diploma (IB).

{\bf Mathematics}, Universitat de Barcelona \hfill {2019 - 2025}\\
\textit{Currently doing}: final thesis 

{\bf Computer Science}, Universitat Politècnica de Catalunya \hfill {2020 - 2025}\\
Specialize in computation. \textit{Pending subjects}: final thesis \\

\end{rSection}

%----------------------------------------------------------------------------------------
% TECHINICAL STRENGTHS	
%----------------------------------------------------------------------------------------
\begin{rSection}{LANGUAGES}

\begin{tabular}{ @{} >{\bfseries}l @{\hspace{6ex}} l }
Catalan, Spanish & Native
\\
English & No difficulties, preparing for the certificate. \\

\end{tabular}\\
\end{rSection}

\begin{rSection}{SKILLS}
I don't have many references to be able to say what level I have, but the languages I have been using these years are (in order of knowledge):\\

\begin{tabular}{ @{} >{\bfseries}l @{\hspace{6ex}} l }
C, C++, Java, Python & hola  \\
Haskell, Rust, Open GL, SQL, Prolog & hola\\
Wolfram Mathematica, R  & hola\\
\end{tabular}\\

I am also used to working with \LaTeX.
\end{rSection}


\begin{rSection}{EXPERIENCE CODING}

\textbf{HackUpc} \hfill  May 2021 and May 2022 
 \begin{itemize}
    \itemsep -3pt {} 
    \item Won the university best app, for developing enFIBat together with 2 friends, using react native. In HackUpc 2022 our team developed a web using react, but I was focused on a rust code project.
 \end{itemize}
\end{rSection} 

\end{document}
